\chapter{Strings and things}

\index{object}
\index{System.in}
\index{System.out}

In Java and other object-oriented languages, an {\bf object} is a collection of data that provides a set of methods.
For example, \java{Scanner}, which we saw in Section~\ref{scanner}, is an object that provides methods for parsing input.
\java{System.out} and \java{System.in} are also objects.

Strings are objects, too.
They contain characters and provide methods for manipulating character data.
We explore some of those methods in this chapter.

\index{primitive}

Not everything in Java is an object: \java{int}, \java{double}, and \java{boolean} are so-called {\bf primitive} types.
We will explain some of the differences between object types and primitive types as we go along.


\section{Strings are immutable}

\index{toUpperCase}
\index{toLowerCase}
\index{immutable}

Strings provide methods, \java{toUpperCase} and \java{toLowerCase}, that convert from uppercase to lowercase and back.
These methods are often a source of confusion, because it sounds like they modify strings.
But neither these methods nor any others can change a string, because strings are {\bf immutable}.

When you invoke \java{toUpperCase} on a string, you get a new string object as a return value.
For example:

\begin{code}
String name = "Alan Turing";
String upperName = name.toUpperCase();
\end{code}

\index{Turing, Alan}

After these statements run, \java{upperName} refers to the string \java{"ALAN TURING"}.
But \java{name} still refers to \java{"Alan Turing"}.

\index{replace}

Another useful method is \java{replace}, which finds and replaces instances of one string within another.
This example replaces \java{"Computer Science"} with \java{"CS"}:

\begin{code}
String text = "Computer Science is fun!";
text = text.replace("Computer Science", "CS");
\end{code}

This example demonstrates a common way to work with string methods.
It invokes \java{text.replace}, which returns a reference to a new string, \java{"CS is fun!"}.
Then it assigns the new string to \java{text}, replacing the old string.

This assignment is important; if you don't save the return value, invoking \java{text.replace} has no effect.

% ABD: Too many new ideas here: the most important one is that you have to do something with the return value.  It's not a good time to appreciate the glory of immutability.

%Strings are immutable by design, because it simplifies passing them between methods as parameters and return values.
%And since the contents of a string can never change, two variables can reference the same string without one accidentally corrupting the other.


\section{Wrapper classes}

Primitive values (like \java{int}s, \java{double}s, and \java{char}s) do not provide methods.
For example, you can't call \java{equals} on an \java{int}:

\begin{code}
int i = 5;
System.out.println(i.equals(5));  // compiler error
\end{code}

\index{wrapper class}
\index{Character}
\index{Integer}
\index{Double}

But for each primitive type, there is a corresponding class in the Java library, called a {\bf wrapper class}.
The wrapper class for \java{char} is called \java{Character}; for \java{int} it's called \java{Integer}.
Other wrapper classes include \java{Boolean}, \java{Long}, and \java{Double}.
They are in the \java{java.lang} package, so you can use them without importing them.

Each wrapper class defines constants \java{MIN_VALUE} and \java{MAX_VALUE}.
For example, \java{Integer.MIN_VALUE} is \java{-2147483648}, and \java{Integer.MAX_VALUE} is \java{2147483647}.
Because these constants are available in wrapper classes, you don't have to remember them, and you don't have to include them in your programs.

Wrapper classes provide methods for converting strings to other types.
For example, \java{Integer.parseInt} converts a string to (you guessed it) an integer:

\begin{code}
String str = "12345";
int num = Integer.parseInt(str);
\end{code}

\index{parse}

In this context, {\bf parse} means something like ``read and translate''.

The other wrapper classes provide similar methods, like \java{Double.parseDouble} and \java{Boolean.parseBoolean}.
They also provide \java{toString}, which returns a string representation of a value:

\begin{code}
int num = 12345;
String str = Integer.toString(num);
\end{code}

The result is the string \java{"12345"}.


\section{Command-line arguments}

Now that you know about arrays and strings, we can {\em finally} explain the \java{args} parameter for \java{main} that we have been ignoring since Chapter~\ref{theway}.
If you are unfamiliar with the command-line interface, please read or review Appendix~\ref{commandline}.

Continuing an earlier example, let's write a program to find the largest value in a sequence of numbers.
Rather than read the numbers from \java{System.in}, we'll pass them as command-line arguments.
Here is a starting point:

\begin{code}
public class Max {
    public static void main(String[] args) {
        System.out.println(Arrays.toString(args));
    }
}
\end{code}

You can run this program from the command line by typing:

\begin{stdout}
java Max
\end{stdout}

\index{empty array}

The output indicates that \java{args} is an {\bf empty array}; that is, it has no elements:

\begin{stdout}
[]
\end{stdout}

But if you provide additional values on the command line, they are passed as arguments to \java{main}.
For example, if you run it like this:

\begin{stdout}
java Max 10 -3 55 0 14
\end{stdout}

The output is:

\begin{stdout}
[10, -3, 55, 0, 14]
\end{stdout}

But remember that the elements of \java{args} are strings.
To find the maximum number, we have to convert the arguments to integers.

The following fragment uses an enhanced \java{for} loop to parse the arguments (using the \java{Integer} wrapper class) and find the largest value:

\begin{code}
int max = Integer.MIN_VALUE;
for (String arg : args) {
    int value = Integer.parseInt(arg);
    if (value > max) {
        max = value;
    }
}
System.out.println("The max is " + max);
\end{code}

The initial value of \java{max} is the smallest (most negative) number an \java{int} can represent, so any other value is greater.
If \java{args} is empty, the result is \java{MIN_VALUE}.


\section{Generating tables}

\index{table}
\index{logarithm}

Loops are good for generating and displaying tabular data.
Before computers were readily available, people had to calculate logarithms, sines and cosines, and other common mathematical functions by hand.
To make that easier, there were books of tables where you could look up values of various functions.
Creating these tables by hand was slow and boring, and the results were often full of errors.

When computers appeared on the scene, one of the initial reactions was: ``This is great!
We can use a computer to generate the tables, so there will be no errors.''
That turned out to be true (mostly), but shortsighted.
Not much later, computers were so pervasive that printed tables became obsolete.

\index{division!floating-point}

Even so, for some operations, computers use tables of values to get an approximate answer, and then perform computations to improve the approximation.
In some cases, there have been errors in the underlying tables, most famously in the table the original Intel Pentium used to perform floating-point division (see \url{https://en.wikipedia.org/wiki/Pentium_FDIV_bug}).

Although a ``log table'' is not as useful as it once was, it still makes a good example of iteration.
The following loop displays a table with a sequence of values in the left column and their logarithms in the right column:

\begin{code}
int i = 1;
while (i < 10) {
    double x = i;
    System.out.println(x + "   " + Math.log(x));
    i = i + 1;
}
\end{code}

The output of this program is:

\begin{stdout}
1.0   0.0
2.0   0.6931471805599453
3.0   1.0986122886681098
4.0   1.3862943611198906
5.0   1.6094379124341003
6.0   1.791759469228055
7.0   1.9459101490553132
8.0   2.0794415416798357
9.0   2.1972245773362196
\end{stdout}

\java{Math.log} computes natural logarithms, that is, logarithms base $e$.
For computer science applications, we often want logarithms with respect to base 2.
To compute them, we can apply this equation:
%
\[ \log_2 x = \frac{log_e x}{log_e 2} \]
%
We can modify the loop as follows:

\begin{code}
int i = 1;
while (i < 10) {
    double x = i;
    System.out.println(x + "   " + Math.log(x) / Math.log(2));
    i = i + 1;
}
\end{code}

And here are the results:

\begin{stdout}
1.0   0.0
2.0   1.0
3.0   1.5849625007211563
4.0   2.0
5.0   2.321928094887362
6.0   2.584962500721156
7.0   2.807354922057604
8.0   3.0
9.0   3.1699250014423126
\end{stdout}

Each time through the loop, we add one to \java{x}, so the result is an arithmetic sequence.
If we multiply \java{x} by something instead, we get a geometric sequence:

\begin{code}
final double LOG2 = Math.log(2);
int i = 1;
while (i < 100) {
    double x = i;
    System.out.println(x + "   " + Math.log(x) / LOG2);
    i = i * 2;
}
\end{code}

\index{final}

The first line stores \java{Math.log(2)} in a \java{final} variable to avoid computing that value over and over again.
The last line multiplies \java{x} by 2.
The result is:

\begin{stdout}
1.0   0.0
2.0   1.0
4.0   2.0
8.0   3.0
16.0   4.0
32.0   5.0
64.0   6.0
\end{stdout}

This table shows the powers of two and their logarithms, base 2.
Log tables may not be useful anymore, but for computer scientists, knowing the powers of two helps a lot!
%When you have an idle moment, you should memorize the powers of two up to 65536 (that's $2^{16}$).


\section{Encapsulation}
\label{encapsulation}

\index{table!two-dimensional}
\index{program development}

In Section~\ref{distance}, we presented a way of writing programs called incremental development.
In this section we present another {\bf program development} process called ``encapsulation and generalization''.
The steps are:

\begin{enumerate}

\item Write a few lines of code in \java{main} or another method, and test them.

\item When they are working, wrap them in a new method, and test again.

\item If it's appropriate, replace literal values with variables and parameters.

\end{enumerate}

\index{encapsulate}
\index{generalize}

The second step is called {\bf encapsulation}; the third step is {\bf generalization}.

To demonstrate this process, we'll develop methods that display multiplication tables.
Here is a loop that displays the multiples of two, all on one line:

\begin{code}
int i = 1;
while (i <= 6) {
    System.out.printf("%4d", 2 * i);
    i = i + 1;
}
System.out.println();
\end{code}

\index{loop variable}
\index{variable!loop}

The first line initializes a variable named \java{i}, which is going to act as a {\bf loop variable}: as the loop executes, the value of \java{i} increases from 1 to 6; when \java{i} is 7, the loop terminates.

Each time through the loop, we display the value \java{2 * i} padded with spaces so it's four characters wide.
Since we use \java{System.out.printf}, the output appears on a single line.

After the loop, we call \java{println} to print a newline and complete the line.
Remember that in some environments, none of the output is displayed until the line is complete.

The output of the code so far is:

\begin{stdout}
   2   4   6   8  10  12
\end{stdout}

The next step is to ``encapsulate'' this code in a new method.
Here's what it looks like:

\begin{code}
public static void printRow() {
    int i = 1;
    while (i <= 6) {
        System.out.printf("%4d", 2 * i);
        i = i + 1;
    }
    System.out.println();
}
\end{code}

\index{generalization}

Next we replace the constant value, \java{2}, with a parameter, \java{n}.
This step is called ``generalization'' because it makes the method more general (less specific).

\begin{code}
public static void printRow(int n) {
    int i = 1;
    while (i <= 6) {
        System.out.printf("%4d", n * i);
        i = i + 1;
    }
    System.out.println();
}
\end{code}

Invoking this method with the argument 2 yields the same output as before.
With the argument 3, the output is:

\begin{stdout}
   3   6   9  12  15  18
\end{stdout}

And with argument 4, the output is:

\begin{stdout}
   4   8  12  16  20  24
\end{stdout}

By now you can probably guess how we are going to display a multiplication table: we'll invoke \java{printRow} repeatedly with different arguments.
In fact, we'll use another loop to iterate through the rows.

\begin{code}
int i = 1;
while (i <= 6) {
    printRow(i);
    i = i + 1;
}
\end{code}

And the output looks like this:

\begin{stdout}
   1   2   3   4   5   6
   2   4   6   8  10  12
   3   6   9  12  15  18
   4   8  12  16  20  24
   5  10  15  20  25  30
   6  12  18  24  30  36
\end{stdout}

The format specifier \java{\%4d} in \java{printRow} causes the output to align vertically, regardless of whether the numbers are one or two digits.

Finally, we encapsulate the second loop in a method:

\begin{code}
public static void printTable() {
    int i = 1;
    while (i <= 6) {
        printRow(i);
        i = i + 1;
    }
}
\end{code}

One of the challenges of programming, especially for beginners, is figuring out how to divide up a program into methods.
The process of encapsulation and generalization allows you to design as you go along.


\section{Generalization}

\index{generalization}

The previous version of \java{printTable} always displays six rows.
We can generalize it by replacing the literal \java{6} with a parameter:

\begin{code}
public static void printTable(int rows) {
    int i = 1;
    while (i <= rows) {
        printRow(i);
        i = i + 1;
    }
}
\end{code}

Here is the output with the argument 7:

\begin{stdout}
   1   2   3   4   5   6
   2   4   6   8  10  12
   3   6   9  12  15  18
   4   8  12  16  20  24
   5  10  15  20  25  30
   6  12  18  24  30  36
   7  14  21  28  35  42
\end{stdout}

That's better, but it still has a problem: it always displays the same number of columns.
We can generalize more by adding a parameter to \java{printRow}:

\begin{code}
public static void printRow(int n, int cols) {
    int i = 1;
    while (i <= cols) {
        System.out.printf("%4d", n * i);
        i = i + 1;
    }
    System.out.println();
}
\end{code}

Now \java{printRow} takes two parameters: \java{n} is the value whose multiples should be displayed, and \java{cols} is the number of columns.
Since we added a parameter to \java{printRow}, we also have to change the line in \java{printTable} where it is invoked:

\begin{code}
public static void printTable(int rows) {
    int i = 1;
    while (i <= rows) {
        printRow(i, rows);
        i = i + 1;
    }
}
\end{code}

When this line executes, it evaluates \java{rows} and passes the value, which is 7 in this example, as an argument.
In \java{printRow}, this value is assigned to \java{cols}.
As a result, the number of columns equals the number of rows, so we get a square 7x7 table:

\begin{stdout}
   1   2   3   4   5   6   7
   2   4   6   8  10  12  14
   3   6   9  12  15  18  21
   4   8  12  16  20  24  28
   5  10  15  20  25  30  35
   6  12  18  24  30  36  42
   7  14  21  28  35  42  49
\end{stdout}

When you generalize a method appropriately, you often find that it has capabilities you did not plan.
For example, you might notice that the multiplication table is symmetric; since $ab = ba$, all the entries in the table appear twice.
You could save ink by printing half of the table, and you would only have to change one line of \java{printTable}:

\begin{code}
printRow(i, i);
\end{code}

In words, the length of each row is the same as its row number.
The result is a triangular multiplication table.

\begin{stdout}
   1
   2   4
   3   6   9
   4   8  12  16
   5  10  15  20  25
   6  12  18  24  30  36
   7  14  21  28  35  42  49
\end{stdout}

Generalization makes code more versatile, more likely to be reused, and sometimes easier to write.

%Even though the second parameter in \java{printRow} is named \java{size} and we have a variable with the same name, we can still use any value or expression we want for the argument.

%Remember, you do not pass {\em variables} to methods; you pass their current {\em values}.
%In this last example, the value of \java{i} in \java{printTable} is assigned to both \java{n} and \java{cols} in \java{printRow}.


\section{The do-while loop}

\index{pretest loop}

The \java{while} and \java{for} statements are {\bf pretest loops}; that is, they test the condition first and at the beginning of each pass through the loop.

\index{posttest loop}
\index{do-while}

Java also provides a {\bf posttest loop}: the \java{do}-\java{while} statement.
This type of loop is useful when you need to run the body of the loop at least once.

%NOTE: can we find an example that's better using do-while than using while-break?

For example, in Section~\ref{validate} we used the \java{return} statement to avoid reading invalid input from the user.
We can use a \java{do}-\java{while} loop to keep reading input until it's valid:

\begin{code}
Scanner in = new Scanner(System.in);
boolean okay;
do {
    System.out.print("Enter a number: ");
    if (in.hasNextDouble()) {
        okay = true;
    } else {
        okay = false;
        String word = in.next();
        System.err.println(word + " is not a number");
    }
} while (!okay);
double x = in.nextDouble();
\end{code}

Although this code looks complicated, it is essentially only three steps:

\begin{enumerate}
\item Display a prompt.
\item Check the input; if invalid, display an error and start over.
\item Read the input.
\end{enumerate}

\index{System.err}

The code uses a flag variable, \java{okay}, to indicate whether we need to repeat the loop body.
If \java{hasNextDouble()} returns \java{false}, we consume the invalid input by calling \java{next()}.
We then display an error message via \java{System.err}.
The loop terminates when \java{hasNextDouble()} return \java{true}.


\section{Break and continue}

Sometimes neither a pretest nor a posttest loop will provide exactly what you need.
In the previous example, the ``test'' needed to happen in the middle of the loop.
As a result, we used a flag variable and a nested \java{if}-\java{else} statement.

\index{break}

A simpler way to solve this problem is to use a \java{break} statement.
When a program reaches a \java{break} statement, it exits the current loop.

\begin{code}
Scanner in = new Scanner(System.in);
while (true) {
    System.out.print("Enter a number: ");
    if (in.hasNextDouble()) {
        break;
    }
    String word = in.next();
    System.err.println(word + " is not a number");
}
double x = in.nextDouble();
\end{code}

Using \java{true} as a conditional in a \java{while} loop is an idiom that means ``loop forever'', or in this case ``loop until you get to a \java{break} statement.''

\index{continue}

In addition to the \java{break} statement, which exits the loop, Java provides a \java{continue} statement that moves on to the next iteration.
For example, the following code reads integers from the keyboard and computes a running total.
The \java{continue} statement causes the program to skip over any negative values.

\begin{code}
Scanner in = new Scanner(System.in);
int x = -1;
int sum = 0;
while (x != 0) {
    x = in.nextInt();
    if (x <= 0) {
        continue;
    }
    System.out.println("Adding " + x);
    sum += x;
}
\end{code}

Although \java{break} and \java{continue} statements give you more control of the loop execution, they can make code difficult to understand and debug.
Use them sparingly.


\section{The switch statement}

TODO


\section{Vocabulary}

\begin{description}

\term{object}
A collection of related data that comes with a set of methods that operate on it.

\term{primitive}
A data type that stores a single value and provides no methods.

%\term{traverse}
%To iterate through the elements of a set performing a similar operation on each.

%\term{index}
%A variable or value used to indicate one of the members of a collection, like a character from a string.

%\term{counter}
%A variable used to count something, usually initialized to zero and then incremented.

%\term{exception}
%A runtime error like ArithmeticException or IndexOutOfBoundsException.

%TODO: find a place to explain how to read a stack trace, ideally with a
% non-trivial call stack (this one is important -- right now we have nothing)

%\term{stack trace}
%An error message that shows the state of a program when an exception occurs.

\term{immutable}
An object that, once created, cannot be modified.
Strings are immutable by design.

%\term{utility method}
%A method that provides commonly needed functionality.

\term{wrapper class}
Classes in \java{java.lang} that provide constants and methods for working with primitive types.

\term{parse}
To read a string and interpret or translate it.

\term{empty array}
An array with no elements and a length of zero.

\term{program development}
A process for writing programs.
So far we have seen ``incremental development'' and ``encapsulation and generalization''.

\term{encapsulate}
To wrap a sequence of statements in a method.

\term{generalize}
To replace something unnecessarily specific (like a constant value) with something appropriately general (like a variable or parameter).

\term{pretest loop}
A loop that tests the condition before each iteration.

\term{posttest loop}
A loop that tests the condition after each iteration.

\end{description}


\section{Exercises}

The code for this chapter is in the {\tt ch09} directory of {\tt ThinkJavaCode2}.
See page~\pageref{code} for instructions on how to download the repository.
Before you start the exercises, we recommend that you compile and run the examples.


\begin{exercise}  %%V6 Ex9.1

The point of this exercise is to explore Java types and fill in some of the details that aren't covered in the chapter.

\index{concatenate}

\begin{enumerate}

\item Create a new program named {\tt Test.java} and write a \java{main} method that contains expressions that combine various types using the \java{+} operator.
For example, what happens when you ``add'' a \java{String} and a \java{char}?
Does it perform character addition or string concatenation?
What is the type of the result?
(How can you determine the type of the result?)

\item Make a bigger copy of the following table and fill it in.
At the intersection of each pair of types, you should indicate whether it is legal to use the \java{+} operator with these types, what operation is performed (addition or concatenation), and what the type of the result is.

\begin{center}
\begin{tabular}{|l|l|l|l|l|l|} \hline
        &  boolean  &  ~char~  &  ~~int~~  &  double  &  String \\ \hline
boolean &           &          &           &          &         \\ \hline
char    &           &          &           &          &         \\ \hline
int     &           &          &           &          &         \\ \hline
double  &           &          &           &          &         \\ \hline
String  &           &          &           &          &         \\ \hline
\end{tabular}
\end{center}

\item Think about some of the choices the designers of Java made when they filled in this table.
How many of the entries seem unavoidable, as if there was no other choice?
How many seem like arbitrary choices from several equally reasonable possibilities?
Which entries seem most problematic?

\item Here's a puzzler: normally, the statement \java{x++} is exactly equivalent to \java{x = x + 1}.
But if \java{x} is a \java{char}, it's not exactly the same!
In that case, \java{x++} is legal, but \java{x = x + 1} causes an error.
Try it out and see what the error message is, then see if you can figure out what is going on.

\item What happens when you add \java{""} (the empty string) to the other types, for example, \java{"" + 5}?

\item For each data type, what types of values can you assign to it?
For example, you can assign an \java{int} to a \java{double} but not vice versa.

\end{enumerate}

\end{exercise}


\begin{exercise}  %%V6 Ex9.2

Write a method called \java{letterHist} that takes a string as a parameter and returns a histogram of the letters in the string.
The zeroth element of the histogram should contain the number of a's in the string (upper- and lowercase); the 25th element should contain the number of z's.
Your solution should only traverse the string once.

\end{exercise}


\begin{exercise}  %%V6 Ex9.3

\index{encapsulation}
\index{generalization}

The purpose of this exercise is to review encapsulation and generalization (see Section~\ref{encapsulation}).
The following code fragment traverses a string and checks whether it has the same number of open and close parentheses:

\begin{code}
String s = "((3 + 7) * 2)";
int count = 0;

for (int i = 0; i < s.length(); i++) {
    char c = s.charAt(i);
    if (c == '(') {
        count++;
    } else if (c == ')') {
        count--;
    }
}

System.out.println(count);
\end{code}

\begin{enumerate}

\item Encapsulate this fragment in a method that takes a string argument and returns the final value of \java{count}.

\item Now that you have generalized the code so that it works on any string, what could you do to generalize it more?

\item Test your method with multiple strings, including some that are balanced and some that are not.

\end{enumerate}

\end{exercise}


\begin{exercise}  %%V6 Ex9.4

Create a program called {\tt Recurse.java} and type in the following methods:

\begin{code}
/**
 * Returns the first character of the given String.
 */
public static char first(String s) {
    return s.charAt(0);
}
\end{code}

\begin{code}
/**
 * Returns all but the first letter of the given String.
 */
public static String rest(String s) {
    return s.substring(1);
}
\end{code}

\begin{code}
/**
 * Returns all but the first and last letter of the String.
 */
public static String middle(String s) {
    return s.substring(1, s.length() - 1);
}
\end{code}

\begin{code}
/**
 * Returns the length of the given String.
 */
public static int length(String s) {
    return s.length();
}
\end{code}

\begin{enumerate}

\item Write some code in \java{main} that tests each of these methods.
Make sure they work, and you understand what they do.

\item Using these methods, and without using any other \java{String} methods, write a method called \java{printString} that takes a string as a parameter and that displays the letters of the string, one on each line.
It should be a void method.

\item Again using only these methods, write a method called \java{printBackward} that does the same thing as \java{printString} but that displays the string backward (again, one character per line).

\item Now write a method called \java{reverseString} that takes a string as a parameter and that returns a new string as a return value.
The new string should contain the same letters as the parameter, but in reverse order.

\begin{code}
String backwards = reverseString("coffee");
System.out.println(backwards);
\end{code}

The output of this example code should be:

\begin{stdout}
eeffoc
\end{stdout}

\index{palindrome}

\item A palindrome is a word that reads the same both forward and backward, like ``otto'' and ``palindromeemordnilap''.
Here's one way to test whether a string is a palindrome:

\begin{quotation}
\noindent
A single letter is a palindrome, a two-letter word is a palindrome if the letters are the same, and any other word is a palindrome if the first letter is the same as the last and the middle is a palindrome.
\end{quotation}

Write a recursive method named \java{isPalindrome} that takes a \java{String} and returns a \java{boolean} indicating whether the word is a palindrome.

\end{enumerate}

\end{exercise}


\begin{exercise}  %%V6 Ex7.5

One way to calculate $e^x$ is to use the infinite series expansion:
%
\[ e^x = 1 + x + x^2 / 2! + x^3 / 3! + x^4 / 4! + \ldots \]
%
The $i$th term in the series is $x^i / i!$.

\begin{enumerate}

\item Write a method called \java{myexp} that takes \java{x} and \java{n} as parameters and estimates $e^x$ by adding the first \java{n} terms of this series.
You can use the \java{factorial} method from Section~\ref{factorial} or your iterative version from the previous exercise.

\index{efficiency}

\item You can make this method more efficient if you realize that the numerator of each term is the same as its predecessor multiplied by \java{x}, and the denominator is the same as its predecessor multiplied by \java{i}.
Use this observation to eliminate the use of \java{Math.pow} and \java{factorial}, and check that you get the same result.

\item Write a method called \java{check} that takes a parameter, \java{x}, and displays \java{x}, \java{myexp(x)}, and \java{Math.exp(x)}.
The output should look something like:

\begin{stdout}
1.0     2.708333333333333     2.718281828459045
\end{stdout}

You can use the escape sequence \java{"\\t"} to put a tab character between columns of a table.

\item Vary the number of terms in the series (the second argument that \java{check} sends to \java{myexp}) and see the effect on the accuracy of the result.
Adjust this value until the estimated value agrees with the correct answer when \java{x} is 1.

\item Write a loop in \java{main} that invokes \java{check} with the values 0.1, 1.0, 10.0, and 100.0.
How does the accuracy of the result vary as \java{x} varies?
Compare the number of digits of agreement rather than the difference between the actual and estimated values.

\item Add a loop in \java{main} that checks \java{myexp} with the values -0.1, -1.0, -10.0, and -100.0.
Comment on the accuracy.

\end{enumerate}

\end{exercise}
