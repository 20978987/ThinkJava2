\usepackage{geometry}
\geometry{
    %paperwidth=6.0in,
    %paperheight=9.0in,
    width=5.5in,
    height=8.5in,
    hmarginratio=1:1,  % 3:2 for binding offset
    vmarginratio=1:1,
    includehead=true,
    headheight=15pt
}

% index with headings
\usepackage{imakeidx}
\makeindex[options= -s headings.ist]

% paragraph spacing
\setlength{\parindent}{0pt}                      % 17.62482pt
\setlength{\parskip}{12pt plus 4pt minus 4pt}    % 0.0pt plus 1.0pt
\linespread{1.05}
\def\arraystretch{1.5}

% list spacing
\setlength{\topsep}{5pt plus 2pt minus 3pt}      % 10.0pt plus 4.0pt minus 6.0pt
\setlength{\partopsep}{-6pt plus 2pt minus 2pt}  %  3.0pt plus 2.0pt minus 2.0pt
\setlength{\itemsep}{0pt}                        %  5.0pt plus 2.5pt minus 1.0pt

% these are copied from tex/latex/base/book.cls
% all I changed is afterskip
\makeatletter
\renewcommand{\section}{\@startsection{section}{1}{\z@}%
    {-3.5ex \@plus -1ex \@minus -.2ex}%
    {0.7ex \@plus.2ex}%
    {\normalfont\Large\bfseries}}
\renewcommand\subsection{\@startsection{subsection}{2}{\z@}%
    {-3.25ex\@plus -1ex \@minus -.2ex}%
    {0.3ex \@plus .2ex}%
    {\normalfont\large\bfseries}}
\renewcommand\subsubsection{\@startsection{subsubsection}{3}{\z@}%
    {-3.25ex\@plus -1ex \@minus -.2ex}%
    {0.3ex \@plus .2ex}%
    {\normalfont\normalsize\bfseries}}
\makeatother

% table of contents vertical spacing
\usepackage{tocloft}
\setlength\cftparskip{8pt plus 4pt minus 4pt}

% balanced index with TOC entry
\usepackage[totoc]{idxlayout}

% The following line adds a little extra space to the column
% in which the Section numbers appear in the table of contents
\makeatletter
\renewcommand{\l@section}{\@dottedtocline{1}{1.5em}{3.0em}}
\makeatother

% customize page headers
\usepackage{fancyhdr}
\pagestyle{fancyplain}
\renewcommand{\chaptermark}[1]{\markboth{Chapter \thechapter ~~ #1}{}}
\renewcommand{\sectionmark}[1]{\markright{\thesection ~~ #1}}
\lhead[\fancyplain{}{\bfseries\thepage}]%
      {\fancyplain{}{\bfseries\rightmark}}
\rhead[\fancyplain{}{\bfseries\leftmark}]%
      {\fancyplain{}{\bfseries\thepage}}
\cfoot{}
%\usepackage[mmddyyyy]{datetime}
%\rfoot{\textcolor{gray}{\tiny \thetitle, \theversion, \today}}

%% tweak spacing of figures and captions
%\usepackage{floatrow}
%\usepackage{caption}
%\captionsetup{
%    font=small,
%    labelformat=empty,
%    justification=centering,
%    skip=4pt
%}

% colors for code listings and output
\usepackage{xcolor}
\definecolor{bgcolor}{HTML}{FAFAFA}
\definecolor{comment}{HTML}{007C00}
\definecolor{keyword}{HTML}{0000FF}
\definecolor{strings}{HTML}{B20000}

% syntax highlighting in code listings
\usepackage{textcomp}
\usepackage{listings}
\lstset{
    language=java,
    basicstyle=\ttfamily,
    backgroundcolor=\color{bgcolor},
    commentstyle=\color{comment},
    keywordstyle=\color{keyword},
    stringstyle=\color{strings},
    columns=fullflexible,
    emph={label},  % keyword?
    keepspaces=true,
    showstringspaces=false,
    upquote=true,
    xleftmargin=15pt,  % \parindent
    framexleftmargin=3pt,
    aboveskip=\parskip,
    belowskip=\parskip
}

% code listing environments
\lstnewenvironment{code}
{\minipage{\linewidth}}
{\endminipage}
\lstnewenvironment{stdout}
{\lstset{commentstyle=,keywordstyle=,stringstyle=}\minipage{\linewidth}}
{\endminipage}

% interactive code listing
\lstnewenvironment{trinket}[2][400]
{\minipage{\linewidth}}
{\endminipage}

% inline syntax formatting
\newcommand{\java}[1]{\lstinline{#1}}%{

% prevent hyphens in names
\hyphenation{DrJava}
\hyphenation{GitHub}
\hyphenation{Javadoc}

% pdf hyperlinks, table of contents, and document properties
\usepackage{hyperref}
\hypersetup{%
  pdftitle={\thetitle: \thesubtitle},
  pdfauthor={\theauthors},
  pdfsubject={\theversion},
  pdfkeywords={},
  bookmarksopen=false,
  bookmarksnumbered=true,
  colorlinks=true,
  citecolor=black,
  filecolor=black,
  linkcolor=black,
  urlcolor=blue
}

% add dot after numbers in pdf bookmarks
\makeatletter
\renewcommand{\Hy@numberline}[1]{#1. }
\makeatother
