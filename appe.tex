\chapter{Extras}
\label{extras}

% TODO: revise this intro and "About the Appendixes" in the preface

This appendix contains previous sections of the book.
We chose to remove them from this edition, because they were not essential to meet the corresponding chapter's goals.
However, you may still find this material useful.

% TODO: Let's talk about whether to keep this appendix.  Sometimes a
% good cut is a good cut :)


\section{Chapter 4}

%%% V6 Section 6.1 (second half)
\subsection*{Unreachable Code}

Sometimes it is convenient to write multiple \java{return} statements, for example, one in each branch of a conditional:

\begin{code}
public static double absoluteValue(double x) {
    if (x < 0) {
        return -x;
    } else {
        return x;
    }
}
\end{code}

Since these \java{return} statements are in a conditional statement, only one will be executed.
As soon as either of them executes, the method terminates without executing any more statements.

\index{dead code}
\index{unreachable}

Code that appears after a \java{return} statement (in the same block), or any place else where it can never be executed, is called {\bf dead code}.
The compiler will give you an ``unreachable statement'' error if part of your code is dead.
For example, this method contains two lines of dead code:

\begin{code}
public static double absoluteValue(double x) {
    if (x < 0) {
        return -x;
        System.out.println("This line is dead.");  // error
    } else {
        return x;
    }
    System.out.println("So is this one.");  // error
}
\end{code}

If you put \java{return} statements inside a conditional statement, you have to make sure that {\em every possible path} through the method reaches a \java{return} statement.
The compiler will let you know if that's not the case.
For example, the following method is incomplete:

\begin{code}
public static double absoluteValue(double x) {
    if (x < 0) {
        return -x;
    } else if (x > 0) {
        return x;
    }
    // error: missing return statement
}
\end{code}

When \java{x} is 0, neither condition is true, so the method ends without hitting a return statement.
The error message in this case might be something like ``missing return statement'', which is confusing since there are already two.

Compiler errors like ``unreachable statement'' and ``missing return statement'' often indicate a problem with your algorithm, not the code.
In the previous example, \java{if (x > 0)} is unnecessary because \java{x} will always be positive or zero at that point.
Changing that \java{else if} to an \java{else} resolves the error.


\begin{description}

\term{dead code}
Part of a program that can never be executed, often because it appears after a \java{return} statement.

\end{description}


%%% V6 Section 6.3 / V6.5 Section 5.7
\subsection*{Method Composition}

\index{composition}

Once you define a method, you can use it to build other methods.
For example, suppose someone gave you two points -- the center of a circle and a point on the perimeter -- and asked for the area of the circle.

Let's say the center point is stored in the variables \java{xc} and \java{yc}, and the perimeter point is in \java{xp} and \java{yp}.
The first step is to find the radius of the circle, which is the distance between the two points.
Fortunately, we have a method that does just that.

\begin{code}
double radius = distance(xc, yc, xp, yp);
\end{code}

The second step is to find the area of a circle with that radius.
We have a method for that computation too.

\begin{code}
double area = calculateArea(radius);
\end{code}

Putting everything together in a new method, we get:

\begin{code}
public static double circleArea
        (double xc, double yc, double xp, double yp) {
    double radius = distance(xc, yc, xp, yp);
    double area = calculateArea(radius);
    return area;
}
\end{code}

Temporary variables like \java{radius} and \java{area} are useful for development and debugging, but once the program is working we can make it more concise by composing the method calls:

\begin{code}
public static double circleArea
        (double xc, double yc, double xp, double yp) {
    return calculateArea(distance(xc, yc, xp, yp));
}
\end{code}

\index{functional decomposition}

This example demonstrates a process known as {\bf functional decomposition}.
We broke a complex computation into simple methods, tested the methods in isolation, and then composed the methods to perform the final computation.
%This process reduces debugging time and yields code that is more likely to be correct and easier to maintain.

%Computer scientists deal with the complexity of large programs by breaking down computations into simpler methods (which in turn may call other methods).
%Data is passed around the program via parameters and return values.


\begin{description}

\term{functional decomposition}
A process for breaking down a complex computation into simple methods, then composing the methods to perform the computation.

\end{description}


%%% V6 Section 6.4 / V6.5 Section 5.8
\subsection*{Overloading Methods}

You might have noticed that \java{circleArea} and \java{calculateArea} perform similar functions.
They both find the area of a circle, but they take different parameters.
For \java{calculateArea}, we have to provide the radius; for \java{circleArea} we provide two points.

\index{overload}

If two methods do the same thing, it is natural to give them the same name.
Having more than one method with the same name is called {\bf overloading}, and it is legal in Java as long as each version of the method takes different parameters.
So we could rename \java{circleArea} to \java{calculateArea}:

\begin{code}
public static double calculateArea
        (double xc, double yc, double xp, double yp) {
    return calculateArea(distance(xc, yc, xp, yp));
}
\end{code}

%Note that this new \java{calculateArea} method is {\em not} recursive.
When you invoke an overloaded method, Java knows which version you want by looking at the arguments that you provide.

\begin{itemize}

\item For \java{calculateArea(3.0)}, Java uses the original \java{calculateArea} method that takes a \java{double} as an argument and interprets it as the radius.

\item For \java{calculateArea(1.0, 2.0, 4.0, 6.0)}, Java uses the new version of \java{calculateArea} (renamed from \java{circleArea}), which interprets the arguments as two points.

\end{itemize}

In fact, the second version of \java{calculateArea} invokes the first version (after invoking the \java{distance} method).

Many Java library methods are overloaded, meaning that there are different versions that accept different numbers or types of parameters.
For example, there are variants of \java{print} and \java{println} that accept a single parameter of any data type.
In the \java{Math} class, there is a version of \java{abs} that works on \java{double}s, and there is also a version for \java{int}s.

Although overloading is a useful feature, it should be used with caution.
You might get yourself nicely confused if you are trying to debug one version of a method while accidentally invoking a different one.


\begin{description}

\term{overload}
To define more than one method with the same name but with different parameters.
%When you invoke an overloaded method, Java knows which version to use by looking at the arguments you provide.

\end{description}


\section{Chapter 6}

%%% V6 Section 7.2
\subsection*{Generating Tables}

\index{table}
\index{logarithm}

%Loops are good for generating and displaying tabular data.
Before computers were readily available, people had to calculate logarithms, sines and cosines, and other common mathematical functions by hand.
To make that easier, there were books of tables where you could look up values of various functions.

Creating these tables by hand was slow and boring, and the results were often full of errors.
When computers appeared on the scene, one of the initial reactions was: ``This is great!
We can use a computer to generate the tables, so there will be no errors.''

That turned out to be true (mostly), but shortsighted.
Not much later, computers were so pervasive that printed tables became obsolete.
Even so, for some operations, computers use tables of values to get an approximate answer, and then perform computations to improve the approximation.

\index{division!floating-point}

In some cases, there have been errors in the underlying tables, most famously in the table the original Intel Pentium used to perform floating-point division (see \url{https://en.wikipedia.org/wiki/Pentium_FDIV_bug}).

Although a ``log table'' is not as useful as it once was, it still makes a good example of iteration.
The following loop displays a table with a sequence of values in the left column and their logarithms in the right column:

\begin{code}
int i = 1;
while (i < 10) {
    double x = i;
    System.out.println(x + "   " + Math.log(x));
    i = i + 1;
}
\end{code}

The output of this program is:

\begin{stdout}
1.0   0.0
2.0   0.6931471805599453
3.0   1.0986122886681098
4.0   1.3862943611198906
5.0   1.6094379124341003
6.0   1.791759469228055
7.0   1.9459101490553132
8.0   2.0794415416798357
9.0   2.1972245773362196
\end{stdout}

\java{Math.log} computes natural logarithms, that is, logarithms base $e$.
For computer science applications, we often want logarithms with respect to base 2.
To compute them, we can apply this equation:
%
\[ \log_2 x = \frac{log_e x}{log_e 2} \]
%
We can modify the loop as follows:

\begin{code}
int i = 1;
while (i < 10) {
    double x = i;
    System.out.println(x + "   " + Math.log(x) / Math.log(2));
    i = i + 1;
}
\end{code}

And here are the results:

\begin{stdout}
1.0   0.0
2.0   1.0
3.0   1.5849625007211563
4.0   2.0
5.0   2.321928094887362
6.0   2.584962500721156
7.0   2.807354922057604
8.0   3.0
9.0   3.1699250014423126
\end{stdout}

Each time through the loop, we add one to \java{x}, so the result is an arithmetic sequence.
If we multiply \java{x} by something instead, we get a geometric sequence:

\begin{code}
final double LOG2 = Math.log(2);
int i = 1;
while (i < 100) {
    double x = i;
    System.out.println(x + "   " + Math.log(x) / LOG2);
    i = i * 2;
}
\end{code}

\index{final}

The first line stores \java{Math.log(2)} in a \java{final} variable to avoid computing that value over and over again.
The last line multiplies \java{x} by 2.
The result is:

\begin{stdout}
1.0   0.0
2.0   1.0
4.0   2.0
8.0   3.0
16.0   4.0
32.0   5.0
64.0   6.0
\end{stdout}

This table shows the powers of two and their logarithms, base 2.
Log tables may not be useful anymore, but for computer scientists, knowing the powers of two helps a lot!
%When you have an idle moment, you should memorize the powers of two up to 65536 (that's $2^{16}$).


%%% V6 Section 7.6
\subsection*{The do-while Loop}

\index{pretest loop}

The \java{while} and \java{for} statements are {\bf pretest loops}; that is, they test the condition first and at the beginning of each pass through the loop.

\index{posttest loop}
\index{do-while}

Java also provides a {\bf posttest loop}: the \java{do}-\java{while} statement.
This type of loop is useful when you need to run the body of the loop at least once.

%NOTE: can we find an example that's better using do-while than using while-break?

For example, in Section~\ref{validate} we used the \java{return} statement to avoid reading invalid input from the user.
We can use a \java{do}-\java{while} loop to keep reading input until it's valid:

\begin{code}
Scanner in = new Scanner(System.in);
boolean okay;
do {
    System.out.print("Enter a number: ");
    if (in.hasNextDouble()) {
        okay = true;
    } else {
        okay = false;
        String word = in.next();
        System.err.println(word + " is not a number");
    }
} while (!okay);
double x = in.nextDouble();
\end{code}

Although this code looks complicated, it is essentially only three steps:

\begin{enumerate}
\item Display a prompt.
\item Check the input; if invalid, display an error and start over.
\item Read the input.
\end{enumerate}

\index{System.err}

The code uses a flag variable, \java{okay}, to indicate whether we need to repeat the loop body.
If \java{hasNextDouble()} returns \java{false}, we consume the invalid input by calling \java{next()}.
We then display an error message via \java{System.err}.
The loop terminates when \java{hasNextDouble()} return \java{true}.


\begin{description}

\term{pretest loop}
A loop that tests the condition before each iteration.

\term{posttest loop}
A loop that tests the condition after each iteration.

\end{description}


%%% V6 Section 7.7
\subsection*{Break and Continue}

Sometimes neither a pretest nor a posttest loop will provide exactly what you need.
In the previous example, the ``test'' needed to happen in the middle of the loop.
As a result, we used a flag variable and a nested \java{if}-\java{else} statement.

\index{break}

A simpler way to solve this problem is to use a \java{break} statement.
When a program reaches a \java{break} statement, it exits the current loop.

\begin{code}
Scanner in = new Scanner(System.in);
while (true) {
    System.out.print("Enter a number: ");
    if (in.hasNextDouble()) {
        break;
    }
    String word = in.next();
    System.err.println(word + " is not a number");
}
double x = in.nextDouble();
\end{code}

Using \java{true} as a conditional in a \java{while} loop is an idiom that means ``loop forever'', or in this case ``loop until you get to a \java{break} statement.''

\index{continue}

In addition to the \java{break} statement, which exits the loop, Java provides a \java{continue} statement that moves on to the next iteration.
For example, the following code reads integers from the keyboard and computes a running total.
The \java{continue} statement causes the program to skip over any negative values.

\begin{code}
Scanner in = new Scanner(System.in);
int x = -1;
int sum = 0;
while (x != 0) {
    x = in.nextInt();
    if (x <= 0) {
        continue;
    }
    System.out.println("Adding " + x);
    sum += x;
}
\end{code}

Although \java{break} and \java{continue} statements give you more control of the loop execution, they can make code difficult to understand.
Use them sparingly.


\section{Chapter 11}

%%% V6 Section 11.9
\subsection*{Modifier Methods}

In Section~\ref{addingtime}, the \java{add} method did not modify the parameters \java{t1} or \java{t2}.
Instead, it created and returned a new \java{Time} object.

Alternatively, we could have written a method like this:

\begin{code}
public void increment(double seconds) {
    this.second += seconds;

    // carry extra seconds
    this.minute += this.second / 60;
    this.second = this.second % 60;

    // carry extra minutes
    this.hour += this.minute / 60;
    this.minute = this.minute % 60;
}
\end{code}

The \java{increment} method modifies an existing \java{Time} object.
It doesn't create a new one, and it doesn't return anything.

\index{modifier method}
\index{method!modifier}

Methods like \java{increment} are called {\bf modifier} methods.
They are usually void, but sometimes they return a reference to the object they modify.

\index{pure method}
\index{method!pure}

In contrast, methods like \java{add} are called {\bf pure} methods, because:

\begin{itemize}

\item They don't modify the parameters.

\item They don't have any other ``side effects'', like printing.

\item The return value only depends on the parameters, not on any other data.

\end{itemize}

Modifiers can be more efficient because they don't create new objects.
But they can also be error-prone.
Especially when objects are aliased, the effects of modifiers can be confusing.

\index{immutable}

If a class provides only getters and pure methods (no setters or modifiers), then the objects will be immutable.
Working with immutable objects can be more difficult at first, but they can save you from long hours of debugging.


\begin{description}

\term{modifier method}
A method that changes the state (instance variables) of an object.

\term{pure method}
A method that depends only on its parameters (including the implicit parameter \java{this}) and no other data.

\end{description}


\section{Chapter 12}

%%% V6 Section 12.10
\subsection*{Recursive Search}

\index{recursion}

Another way to implement binary search is with a recursive method.
We take \java{low} and \java{high} as parameters, add a base case, and turn steps 3 and 4 into recursive invocations.
%They indicate the segment of the array that should be searched (including both \java{low} and \java{high}).
%Here's what that code looks like:

\begin{code}
public static int binarySearch(Card[] cards, Card target,
                               int low, int high) {
    if (high < low) {
        return -1;
    }
    int mid = (low + high) / 2;                     // step 1
    int comp = cards[mid].compareTo(target);

    if (comp == 0) {                                // step 2
        return mid;
    } else if (comp < 0) {                          // step 3
        return binarySearch(cards, target, mid + 1, high);
    } else {                                        // step 4
        return binarySearch(cards, target, low, mid - 1);
    }
}
\end{code}

Instead of a \java{while} loop, we have an \java{if} statement to terminate the recursion.
%We call this \java{if} statement the base case.
If \java{high} is less than \java{low}, there are no cards between them, and we conclude that the card is not in the array.

\index{infinite recursion}
\index{recursion!infinite}
\index{StackOverflowError}
\index{exception!StackOverflow}

Two common errors in recursive methods are (1) forgetting to include a base case, and (2) writing the recursive call so that the base case is never reached.
Either error causes infinite recursion and a \java{StackOverflowError}.


\section{Chapter 15}
\label{app:langton}

In Chapter~\ref{conway}, we developed classes to implement Conway's Game of Life.
As an exercise, you had a chance to implement Langton's Ant.

We now present a solution to that exercise and use it to demonstrate more advanced object-oriented techniques.


\subsection*{Langton's Ant Solution}

We begin by defining a \java{Langton} class that ``has~a'' grid and information about the ant.
The constructor takes the grid dimensions as parameters.

\begin{code}
public class Langton {
    private GridCanvas grid;
    private int xpos;
    private int ypos;
    private int head; // 0=North, 1=East, 2=South, 3=West

    public Langton(int rows, int cols) {
        grid = new GridCanvas(rows, cols, 10);
        xpos = rows / 2;
        ypos = cols / 2;
        head = 0;
    }
}
\end{code}

\java{grid} is a \java{GridCanvas} object, which represents the state of the cells.
\java{xpos} and \java{ypos} are the coordinates of the ant, and \java{head} is the ``heading'' of the ant, that is, what direction it is facing.
\java{head} is an integer with four possible values, where 0 means the ant is facing ``north'', that is, toward the top of the screen, 1 means ``east'', etc.

Here's the \java{update} method that implements the rules for Langton's ant:

\begin{code}
public void update() {
    flipCell();
    moveAnt();
}
\end{code}

The \java{flipCell} method gets the \java{Cell} at the ant's location, figures out which way to turn, and changes the state of the cell.

\begin{code}
private void flipCell() {
    Cell cell = grid.getCell(xpos, ypos);
    if (cell.isOff()) {
        // at a white square; turn right and flip color
        head = (head + 1) % 4;
        cell.turnOn();
    } else {
        // at a black square; turn left and flip color
        head = (head + 3) % 4;
        cell.turnOff();
    }
}
\end{code}

We use the remainder operator, \verb"%", to make \java{head} wrap around: if \java{head} is 3 and we turn right, it wraps around to 0; if \java{head} is 0 and we turn left, it wraps around to 3.

Notice that to turn right, we add 1 to \java{head}.
To turn left, we could subtract 1, but \verb"-1 % 4" is \verb"-1" in Java.
So we add 3 instead, since one left turn is the same as three right turns.

The \java{moveAnt} method moves the ant forward one square, using \java{head} to determine which way is forward.

\begin{code}
private void moveAnt() {
    if (head == 0) {
        ypos -= 1;
    } else if (head == 1) {
        xpos += 1;
    } else if (head == 2) {
        ypos += 1;
    } else {
        xpos -= 1;
    }
}
\end{code}

If you run this code with a grid size of 61x61 or larger, you will see the ant eventually settle into a repeating pattern.

Here is the \java{main} method we use to create and display the \java{Langton} object:

\begin{code}
public static void main(String[] args) {
    String title = "Langton's Ant";
    Langton game = new Langton(61, 61);
    JFrame frame = new JFrame(title);
    frame.setDefaultCloseOperation(JFrame.EXIT_ON_CLOSE);
    frame.setResizable(false);
    frame.add(game.grid);
    frame.pack();
    frame.setVisible(true);
    game.mainloop();
}
\end{code}

Most of this method is the same as the \java{main} we used to create and run \java{Conway}, in Section~\ref{conwaymain}.
It creates and configures a \java{JFrame} and runs \java{mainloop}.


\subsection*{Refactoring}

Whenever you see repeated code like \java{main}, you should think about ways to remove it.
In the Chapter~\ref{eights}, we used inheritance to eliminate repeated code.
We'll do something similar with \java{Conway} and \java{Langton}.

First, we define a superclass named \java{Automaton} where we will put the code that \java{Conway} and \java{Langton} have in common.

\begin{code}
public class Automaton {
    private GridCanvas grid;

    public void run(String title, int rate) {
        JFrame frame = new JFrame(title);
        frame.setDefaultCloseOperation(JFrame.EXIT_ON_CLOSE);
        frame.setResizable(false);
        frame.add(this.grid);
        frame.pack();
        frame.setVisible(true);
        this.mainloop(rate);
    }
}
\end{code}

\java{Automaton} declares \java{grid} as an instance variable, so every \java{Automaton} ``has~a'' \java{GridCanvas}.
It also provides \java{run}, which contains the code that creates and configures the \java{JFrame}.

The \java{run} method takes two parameters: the window \java{title} and the frame \java{rate}, that is, the number of time steps to show per second.
It uses \java{title} when creating the \java{JFrame}, and it passes \java{rate} to \java{mainloop}:

\begin{code}
private void mainloop(int rate) {
    while (true) {

        // update the drawing
        this.update();
        grid.repaint();

        // delay the simulation
        try {
            Thread.sleep(1000 / rate);
        } catch (InterruptedException e) {
            // do nothing
        }
    }
}
\end{code}

\java{mainloop} contains the code we first saw in Section~\ref{mainloop}.
It runs a \java{while} loop forever (or until the window closes).
Each time through the loop, it runs \java{update} to update \java{grid} and then \java{repaint} to redraw the grid.

Then it calls \java{Thread.sleep} with a delay that depends on \java{rate}.
For example, if {\tt rate} is 2, we should draw two frames per second, so the delay is a half second, or 500 milliseconds.

\index{refactoring}

This process of reorganizing existing code, without changing its behavior, is known as {\bf refactoring}.
We're almost finished; we just need to redesign \java{Conway} and \java{Langton} to extend \java{Automaton}.


\subsection*{Abstract Classes}

So far the implementation works, and if we are not planning to implement any other zero-person games, we could leave well enough alone.
But there are a few problems with the current design:

\begin{enumerate}

\item The \java{grid} attribute is \java{private}, making it inaccessible in \java{Conway} and \java{Langton}.
We could make it \java{public}, but then other (unrelated) classes would have access to it as well.

\item The \java{Automaton} class has no constructors, and even if it did, there would be no reason to create an instance of this class.

\item The \java{Automaton} class does not provide an implementation of \java{update}.
In order to work properly, subclasses need to provide one.

\end{enumerate}

\index{protected}
\index{abstract}

Java provides language features to solve these problems:

\begin{enumerate}

\item We can make the \java{grid} attribute \java{protected}, which means it's accessible to subclasses but not other classes.

\item We can make the class \java{abstract}, which means it cannot be instantiated.
If you attempt to create an object for an abstract class, you will get a compiler error.

\item We can declare \java{update} as an \java{abstract} method, meaning that it must be overriden in subclasses.
If the subclass does not override an abstract method, you will get a compiler error.
\end{enumerate}

Here's what \java{Automaton} looks like as an abstract class:

\begin{code}
public abstract class Automaton {
    protected GridCanvas grid;

    public abstract void update();

    public void run(String title, int delay) {
        // body of this method omitted
    }

    private void mainloop(int rate) {
        // body of this method omitted
    }
}
\end{code}

Notice that the \java{update} method has no body.
The declaration specifies the name, arguments, and return type.
But it does not provide an implementation, because it is an abstract method.

Notice also the word \java{abstract} on the first line, which declares that \java{Automaton} is an abstract class.
In order to have any abstract methods, a class must be declared as abstract.

Any class that extends \java{Automaton} must provide an implementation of \java{update}; the declaration here allows the compiler to check.

Here's what \java{Conway} looks like as a subclass of \java{Automaton}:

\begin{code}
public class Conway extends Automaton {

    // same methods as before, except mainloop is removed

    public static void main(String[] args) {
        String title = "Conway's Game of Life";
        Conway game = new Conway("pulsar.cells", 2);
        game.run(title, 2);
    }
}
\end{code}

\java{Conway} extends \java{Automaton}, so it inherits the \java{protected} instance variable \java{grid} and the methods \java{run} and \java{mainloop}.
But because \java{Automaton} is abstract, \java{Conway} has to provide \java{update} and a constructor (which it has already).

Abstract classes are essentially ``incomplete'' class definitions that specify methods to be implemented by subclasses.
But they also provide attributes and methods to be inherited, thus eliminating repeated code.


%\section{Vocabulary}

\begin{description}

\term{refactor}
The process of restructuring or reorganizing existing code without changing its behavior.

\term{abstract class}
A class that is declared as \java{abstract}; it cannot be instantiated, and it may (or may not) include abstract methods.

\term{concrete class}
A class that is {\em not} declared as \java{abstract}; each of its methods must have an implementation.

\end{description}


%\section{Exercises}

\begin{exercise}

The last section of this appendix introduced \java{Automaton} as an abstract class and rewrote \java{Conway} as a subclass of \java{Automaton}.
Now it's your turn: rewrite \java{Langton} as a subclass of \java{Automaton}, removing the code that's no longer needed.

\end{exercise}


\begin{exercise}

Mathematically speaking, Game of Life and Langton's Ant are cellular automata, where ``cellular'' means it has cells, and ``automaton'' means it runs itself.
See \url{https://en.wikipedia.org/wiki/Cellular_automaton} for more discussion.

Implement another cellular automaton of your choice.
You may have to modify \java{Cell} and/or \java{GridCanvas}, in addition to extending \java{Automaton}.
For example, Brian's Brain (\url{https://en.wikipedia.org/wiki/Brian's_Brain}) requires three states: ``on'', ``dying'', and ``off''.

\end{exercise}
